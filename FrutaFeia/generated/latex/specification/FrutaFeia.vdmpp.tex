\begin{vdmpp}[breaklines=true]
class FrutaFeia

instance variables
 public centros : set of CentroDistribuicao;
 public agricultores: set of Agricultor;
 
 inv forall centro in set centros & exists1 cent in set centros & centro.localizacao = cent.localizacao;
(*@
\label{FrutaFeia:8}
@*)
 inv forall agricultor in set agricultores & exists1 agro in set agricultores & agro.nome = agricultor.nome;
 inv forall centro in set centros & forall cli in set centro.clientes & exists1 cliente in set centro.clientes & cli.nome = cliente.nome;
   
 
(*@
\label{adicionaCliente:12}
@*)
operations 
 public FrutaFeia: () ==> FrutaFeia
 FrutaFeia() == (centros := {}; agricultores := {})
 post centros = {} and agricultores = {};

 --Adiciona cliente ao centro respectivo
 public adicionaCliente: Cliente * CentroDistribuicao`Local ==> ()
(*@
\label{removeCliente:19}
@*)
 adicionaCliente( cliente, centroLocal ) == 
 for all centro in set centros 
 do
   if ( centro.localizacao = centroLocal)
    then  centro.adicionaCliente(cliente)
 pre forall centro in set centros & (forall cli in set centro.clientes & cli.nome <> cliente.nome)
 post exists1 centro in set centros & cliente in set centro.clientes;  
(*@
\label{adicionaCentro:26}
@*)
 
 -- Remove cliente do seu centro respectivo
 public removeCliente:  Cliente`Nome * CentroDistribuicao`Local ==> ()  
 removeCliente(clienteNome, centroLocal) == 
  for all centro in set centros 
 do
(*@
\label{removeCentro:32}
@*)
   if ( centro.localizacao = centroLocal)
    then  centro.removeCliente(clienteNome)
 pre exists1 centro in set centros & (exists1 cliente in set centro.clientes & cliente.nome = clienteNome)
 post forall  centro in set centros & (forall cliente in set centro.clientes & cliente.nome (*@\vdmnotcovered{<>}@*) (*@\vdmnotcovered{clienteNome}@*));  
 
 -- adiciona um centro de Distribuicao do registo
(*@
\label{adicionaAgricultor:38}
@*)
 public adicionaCentro: CentroDistribuicao  ==> ()
 adicionaCentro(centro) ==
   centros := centros union {centro}
 pre centros <> undefined and forall cent in set centros & centro.(*@\vdmnotcovered{localizacao}@*) (*@\vdmnotcovered{<>}@*) cent.(*@\vdmnotcovered{localizacao}@*)
 post card centros = card centros~ +1 and exists1 cent in set centros &  centro.localizacao= cent.localizacao;
 
(*@
\label{removeAgricultor:44}
@*)
 -- remove um Centro de Distribui�ao do registo
 public removeCentro: CentroDistribuicao`Local  ==> ()
 removeCentro(centroLocal) ==
  for all centro in set centros
  do
   if(centroLocal = centro.localizacao)
    then (centros := centros \ {centro}; return)
(*@
\label{geraCestas:51}
@*)
 pre centros <> undefined and exists1 centro in set centros & centro.localizacao = centroLocal
 post card centros = card centros~ - 1  and forall centro in set centros & (*@\vdmnotcovered{centro}@*).(*@\vdmnotcovered{localizacao}@*) <> (*@\vdmnotcovered{centroLocal}@*);
 
 -- adiciona um agricultor do registo
(*@
\label{getTodosClientes:55}
@*)
 public adicionaAgricultor: Agricultor ==> ()
 adicionaAgricultor(agricultor) ==
   agricultores := agricultores union {agricultor}
 pre agricultor <> undefined and agricultor not in set agricultores
 post card agricultores = card agricultores~ +1 ;
 
  -- retira um agricultor registo 
 public removeAgricultor: Agricultor`Nome  ==> ()
 removeAgricultor(agricultorNome) ==
 for all agricultor in set agricultores
  do 
   if( agricultor.nome = agricultorNome)
(*@
\label{getTodosProdutos:67}
@*)
   then(agricultores := agricultores \ {agricultor}; return;) 
 pre agricultores <> undefined and exists1 agricultor in set agricultores & agricultor.nome = agricultorNome
 post card agricultores = card agricultores~ - 1 and forall agricultor in set agricultores & agricultor.nome (*@\vdmnotcovered{<>}@*) (*@\vdmnotcovered{agricultorNome}@*) ;
 
 -- Retorna todos os clientes distribuidos pelos diferentes centros
 public getTodosClientes: () ==> set of Cliente
 getTodosClientes() == 
 (
  dcl  clientes: set of Cliente := {};
  for all centro in set centros 
   do 
    clientes := clientes union centro.clientes;
(*@
\label{geraCestaGrande:79}
@*)
  return clientes;
 )
 pre centros <> undefined;
 
 
 -- Retorna todos os produtos de todos os agricultores ( os produtos com o mesmo nome de diferentes agricultores sao indepedentes)
 public getTodosProdutos: () ==> set of Produto
(*@
\label{geraCestaPequena:86}
@*)
 getTodosProdutos () ==
 (
  dcl produtos: set of Produto := {};
  
  for all agricultor in set agricultores
   do
    produtos := produtos union rng agricultor.stock;
(*@
\label{geraCesta:93}
@*)
  return produtos;
 )
 pre agricultores <> undefined;
 
 --Gera uma cesta grande
 public geraCestaGrande: () ==> Cesta
 geraCestaGrande() == 
  (
   dcl cesta : Cesta := geraCesta(6, 0.75, 8, <GRANDE> );
   return cesta;
  );
 
 --Gera uma cesta pequena 
 public geraCestaPequena: () ==> Cesta
 geraCestaPequena() == 
  ( 
   dcl cesta : Cesta := geraCesta(3, 0.430, 7, <PEQUENA>);
   return cesta
  );
 
 --Cria uma cesta com determinados parametros
 public geraCesta: real * real * real * Cesta`Tamanho==> Cesta
 geraCesta(pesoMinimo, pesoMinimoProduto, totalProdutos, tipoCesta) ==
 (
  dcl cesta : Cesta := new Cesta();
  dcl produtos : set of Produto := getTodosProdutos();
  dcl totalNaCesta : nat := 0;
  dcl pesoAretirar : real := pesoMinimoProduto;
  dcl minimo: real := pesoMinimo;
  dcl runs: nat := 0;
  
  cesta.alterarTamanho(tipoCesta);
  
  --Adiciona produtos � cesta
  while(minimo > 0 and runs < 3)
  do
   (
    for all produto in set produtos
    do 
     if(produto.peso >= pesoAretirar)
      then 
      ( 
      if(cesta.produtoNaCesta(produto.nome) = false and totalNaCesta < totalProdutos) --Se o produto nao esta na cesta e o numero minimo de produtos nao foi alcan�ado
       then
        ( produto.removePeso(pesoAretirar);
          cesta.adicionaProduto(new Produto(produto.nome, produto.origem, pesoAretirar));
          totalNaCesta := totalNaCesta + 1;
(*@
\label{preencheCesta:140}
@*)
          minimo := minimo - pesoAretirar;
        )
       else 
       (
        if(totalNaCesta = totalProdutos  and minimo <= 0) -- Se a cesta esta completa
          then return cesta
          else 
          ( --Adiciona mais X peso a um produto numa cesta
           if(cesta.produtoNaCestaPeso(produto.nome) + pesoAretirar < 0.8) -- limite de +- 50% que um produto pode ocupar
           then
            ( produto.removePeso(pesoAretirar); 
             if(cesta.adicionaPesoProduto(produto.nome, pesoAretirar))
              then minimo := minimo - pesoAretirar
             else produto.adicionaPeso((*@\vdmnotcovered{pesoAretirar}@*))
            );
          );
       );
      );
    pesoAretirar := pesoAretirar / (runs + 1);
    runs := runs + 1;
   );
   return cesta;
 ); 
 
 public preencheCesta: Cesta ==> () -- procura produtos e tenta responder as necessidades da cesta se houver stock
 preencheCesta(cesta) == 
 (
  dcl produtos : set of Produto := getTodosProdutos();
  dcl totalNaCesta : nat := card cesta.produtos;
  dcl pesoAretirar : real := 0.3;
  dcl minimo: real := 0;
  dcl runs: nat := 0;
  dcl totalProdutos: nat := 0;
  
  --- setup com os requesitos das cestas
  if(cesta.tamanho = <PEQUENA>)
    then (
       totalProdutos := 7;
       minimo := 3 - cesta.peso;
      )
  else 
   if(cesta.tamanho = <GRANDE>)
    then ( totalProdutos := 8;
        minimo := 6 - cesta.peso;
       );
     
  --Adicionar produtos � cesta
   while(minimo > 0 and runs < 3)
  do
   (
    for all produto in set produtos
    do 
     if(produto.peso >= pesoAretirar)
      then 
(*@
\label{geraCestaTodosClientes:194}
@*)
      ( 
      if(cesta.produtoNaCesta(produto.nome) = false and totalNaCesta < totalProdutos) -- Se nao existir na cesta e a cesta nao tem o numero de produtos minimo
       then
        ( 
          -- adicionar produto � cesta
          produto.removePeso(pesoAretirar);
          cesta.adicionaProduto(new Produto(produto.nome, produto.origem, pesoAretirar));
          totalNaCesta := totalNaCesta + 1;
          minimo := minimo - pesoAretirar; 
        )
       else 
       (
        if(totalNaCesta = totalProdutos  and minimo <= 0) -- Se a encomenda ja esta completa
          then return
          else 
          ( -- tenta adicionar X peso de um produto que ja se encontra na cesta
           if(cesta.produtoNaCestaPeso(produto.nome) + pesoAretirar < 0.8) 
           then (
            produto.removePeso(pesoAretirar);
            if(cesta.adicionaPesoProduto(produto.nome, pesoAretirar))
              then minimo := minimo - pesoAretirar
             else (*@\vdmnotcovered{produto}@*).adicionaPeso((*@\vdmnotcovered{pesoAretirar}@*));
         );
          );
       );
      );
    pesoAretirar := pesoAretirar / (runs + 1); -- tenta adicionar quantidades mais pequenas 
    runs := runs + 1;
   );
 );
 
 public geraCestaTodosClientes: () ==> ()
 geraCestaTodosClientes() == 
  for all centro in set centros  
  do (
   for all cliente in set centro.clientes -- todos os clientes em todos os centros
   do (
    IO`println(cliente);
    if cliente.estadoEnc = <COM_ENC> -- tenta corresponder a uma encomenda feita pelo cliente
     then (
         if (cliente.encomenda.tamanho = <PEQUENA>) 
          then ( cliente.mudaCesta(geraCestaPequena());
              cliente.verificaCestaParametros(); -- avalia se os requesitos da encomenda estao cumpridoss 
            )
        else
         if(cliente.encomenda.tamanho = <GRANDE>)
          then( cliente.mudaCesta(geraCestaGrande());
             cliente.verificaCestaParametros()  
            )
        )    
    else ( if cliente.estadoEnc = <POR_CONCLUIR>  -- tenta completar uma encomenda pendente
        then (
           preencheCesta(cliente.encomenda);
           cliente.verificaCestaParametros();
            )
    );
    );    
   );
end FrutaFeia




\end{vdmpp}
\bigskip
\begin{longtable}{|l|r|r|r|}
\hline
Function or operation & Line & Coverage & Calls \\
\hline
\hline
\hyperref[FrutaFeia:8]{FrutaFeia} & 8&100.0\% & 6 \\
\hline
\hyperref[adicionaAgricultor:38]{adicionaAgricultor} & 38&100.0\% & 10 \\
\hline
\hyperref[adicionaCentro:26]{adicionaCentro} & 26&83.8\% & 4 \\
\hline
\hyperref[adicionaCliente:12]{adicionaCliente} & 12&100.0\% & 8 \\
\hline
\hyperref[geraCesta:93]{geraCesta} & 93&98.0\% & 10 \\
\hline
\hyperref[geraCestaGrande:79]{geraCestaGrande} & 79&100.0\% & 4 \\
\hline
\hyperref[geraCestaPequena:86]{geraCestaPequena} & 86&100.0\% & 6 \\
\hline
\hyperref[geraCestaTodosClientes:194]{geraCestaTodosClientes} & 194&100.0\% & 2 \\
\hline
\hyperref[geraCestas:51]{geraCestas} & 51&82.6\% & 2 \\
\hline
\hyperref[getTodosClientes:55]{getTodosClientes} & 55&100.0\% & 6 \\
\hline
\hyperref[getTodosProdutos:67]{getTodosProdutos} & 67&100.0\% & 16 \\
\hline
\hyperref[preencheCesta:140]{preencheCesta} & 140&98.4\% & 4 \\
\hline
\hyperref[removeAgricultor:44]{removeAgricultor} & 44&89.4\% & 0 \\
\hline
\hyperref[removeCentro:32]{removeCentro} & 32&89.4\% & 2 \\
\hline
\hyperref[removeCliente:19]{removeCliente} & 19&85.1\% & 4 \\
\hline
\hline
FrutaFeia.vdmpp & & 96.1\% & 84 \\
\hline
\end{longtable}

