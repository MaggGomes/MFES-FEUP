\begin{vdmpp}[breaklines=true]
class TestFrutaFeia is subclass of SuiteTestCase

operations 

(*@
\label{testGeraCestaPequena:5}
@*)
(*@
\label{testAdicionaRemove:5}
@*)
public (*@\vdmnotcovered{testAdicionaRemove}@*)() == (
 dcl frutaFeia : FrutaFeia := new FrutaFeia(); 
 dcl agricultor: Agricultor := new Agricultor("Pedro", "Mirandela");
 dcl cliente: Cliente := new Cliente("Fernando", <HOMEM>);
 dcl centro: CentroDistribuicao := new CentroDistribuicao("Porto");
 IO`println("\t Testing AdicionaRemove");
 assertEqual(0, card frutaFeia.agricultores);  
 frutaFeia.adicionaAgricultor(agricultor);
 assertEqual(1, card frutaFeia.agricultores);
 frutaFeia.removeAgricultor(agricultor);
 assertEqual(0, card frutaFeia.agricultores);
 assertEqual(0, card frutaFeia.centros);
 frutaFeia.adicionaCentro(centro);
 assertEqual(1, card frutaFeia.centros);
 assertEqual(0, card frutaFeia.getTodosClientes());
 frutaFeia.adicionaCliente(cliente, "Porto");
 assertEqual(1, card frutaFeia.getTodosClientes());
 frutaFeia.removeCliente(cliente, "Porto");
 assertEqual(0, card frutaFeia.getTodosClientes());
 frutaFeia.removeCentro(centro);
 assertEqual(0, card frutaFeia.centros);
);

public (*@\vdmnotcovered{testGeraCestaPequena}@*)() == (
 dcl frutaFeia : FrutaFeia := new FrutaFeia(); 
 dcl agricultor: Agricultor := new Agricultor("Pedro", "Mirandela"); 
 dcl agricultor2: Agricultor := new Agricultor("Manuel", "Porto"); 
 dcl pera: Produto := new Produto("Pera", "Mate", 2);
 dcl manga: Produto := new Produto("Manga", "Rosa", 3);
 dcl ananas: Produto := new Produto("Ananas", "Preto", 4);
 dcl maca: Produto := new Produto("Maca", "Golden", 5);
 dcl maracuja: Produto := new Produto("Maracuja", "Bambuin", 6);
  dcl alface: Produto := new Produto("Alface", "Alfacinha", 7);
  dcl pimento: Produto := new Produto("Pimento", "Verde", 8);
  dcl laranja: Produto := new Produto("Laranja", "Algarve", 8);
 dcl cesta: Cesta;
 IO`println("\t Testing GeraCestaPequena"); 
(*@
\label{testeGereTodasEncomendas:42}
@*)
 
 agricultor.adicionaProduto(pera);
 agricultor.adicionaProduto(manga);
 agricultor.adicionaProduto(ananas);
 
 agricultor2.adicionaProduto(maca);
 agricultor2.adicionaProduto(maracuja);
 agricultor2.adicionaProduto(alface);
 agricultor2.adicionaProduto(pimento);
 agricultor2.adicionaProduto(laranja);
 
 frutaFeia.adicionaAgricultor(agricultor);
 frutaFeia.adicionaAgricultor(agricultor2);
 
 IO`println("\t\t Cesta:" );
 cesta := frutaFeia.geraCestaPequena();
 frutaFeia.preencheCesta(cesta);
 assertTrue(cesta.peso > 3);
 IO`println(cesta); 
 assertEqual(7, card cesta.produtos);
 cesta := frutaFeia.geraCestaGrande();
 assertEqual(<GRANDE>, cesta.tamanho);
 frutaFeia.preencheCesta(cesta);
 assertTrue(cesta.peso > 6);
);

public (*@\vdmnotcovered{testeGereTodasEncomendas}@*)() == (
 dcl frutaFeia : FrutaFeia := new FrutaFeia(); 
 dcl agricultor: Agricultor := new Agricultor("Pedro", "Mirandela"); 
 dcl agricultor2: Agricultor := new Agricultor("Manuel", "Porto");
 dcl pera: Produto := new Produto("Pera", "Mate", 2);
 dcl manga: Produto := new Produto("Manga", "Rosa", 3);
 dcl ananas: Produto := new Produto("Ananas", "Preto", 4);
 dcl maca: Produto := new Produto("Maca", "Golden", 5);
 dcl maracuja: Produto := new Produto("Maracuja", "Bambuin", 6);
  dcl alface: Produto := new Produto("Alface", "Alfacinha", 7);
  dcl pimento: Produto := new Produto("Pimento", "Verde", 8);
  dcl laranja: Produto := new Produto("Laranja", "Algarve", 8);  
  dcl centro: CentroDistribuicao := new CentroDistribuicao("Porto");
  dcl cliente: Cliente := new Cliente("Fernando", <HOMEM>);
  dcl cliente1: Cliente := new Cliente("Alfredo", <HOMEM>);
  dcl cliente2: Cliente := new Cliente("Joana", <MULHER>);
  dcl cesta: Cesta := new Cesta(); 
  dcl cesta1: Cesta := new Cesta(); 
 IO`println("\t Testing GereTodasEncomendas:" );  
  
 agricultor.adicionaProduto(pera);
 agricultor.adicionaProduto(manga);
 agricultor.adicionaProduto(ananas);
 
 agricultor2.adicionaProduto(maca);
 agricultor2.adicionaProduto(maracuja);
 agricultor2.adicionaProduto(alface);
(*@
\label{testAll:95}
@*)
 agricultor2.adicionaProduto(pimento);
 agricultor2.adicionaProduto(laranja);
  
 frutaFeia.adicionaAgricultor(agricultor);
 frutaFeia.adicionaAgricultor(agricultor2);
 
 frutaFeia.adicionaCentro(centro);
 frutaFeia.adicionaCliente(cliente, "Porto");
 frutaFeia.adicionaCliente(cliente1, "Porto");
 frutaFeia.adicionaCliente(cliente2, "Porto");
 
 cliente.mudaCesta(new Cesta());
 cesta.alterarTamanho(<GRANDE>);
 cliente1.mudaCesta(cesta);
 cliente2.mudaCesta(cesta1);
 cliente2.estadoEnc := <POR_CONCLUIR>;
 frutaFeia.geraCestaTodosClientes(); 
 IO`println("\t\t Cliente Encomenda:" );
 IO`println(cliente.encomenda);
 IO`println(card centro.clientes);
 IO`println(centro.clientes); 
);

public (*@\vdmnotcovered{testAll}@*)() == (
 testAdicionaRemove();
 testGeraCestaPequena();
 testeGereTodasEncomendas();
);

end TestFrutaFeia
\end{vdmpp}
\bigskip
\begin{longtable}{|l|r|r|r|}
\hline
Function or operation & Line & Coverage & Calls \\
\hline
\hline
\hyperref[testAdicionaRemove:5]{testAdicionaRemove} & 5&98.6\% & 8 \\
\hline
\hyperref[testAll:95]{testAll} & 95&80.0\% & 1 \\
\hline
\hyperref[testGeraCestaPequena:5]{testGeraCestaPequena} & 5&99.0\% & 6 \\
\hline
\hyperref[testeGereTodasEncomendas:42]{testeGereTodasEncomendas} & 42&99.2\% & 3 \\
\hline
\hline
TestFrutaFeia.vdmpp & & 98.7\% & 18 \\
\hline
\end{longtable}

